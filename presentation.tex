\documentclass{beamer}
\usepackage{kotex}
\usepackage{amsmath}
\usepackage{url}

\title{중간발표}
\author{장준민 \and 장지훈 \and 이예담}
\institute{KMU EE}
\date{\today}
\usetheme{Madrid}

\begin{document}
\frame{\titlepage}

\begin{frame}
\frametitle{Constraints we kept}
    \begin{block}{About Vehicle}
        \setlength{\abovedisplayskip}{0pt}
        \setlength{\belowdisplayskip}{0pt}
        \begin{align*}
            (0 < VehicleSpeed \leq 5.5) \; &\wedge \\
            (InitialSpeed = 5.5) \; &\wedge \\
            UseAllowedSensor \; &\wedge \\
        \end{align*}
    \end{block}
    \begin{block}{About Task}
        \setlength{\abovedisplayskip}{0pt}
        \setlength{\belowdisplayskip}{0pt}
        \begin{align*}
            (DetectChild \implies Stop) \; &\wedge \\ 
            (DetectRedLight \implies Stop) \; &\wedge \\
            LaneKeeping\; &\wedge \\
            Parking \; &\wedge \\
        \end{align*}
    \end{block}
\end{frame}

\begin{frame}
\frametitle{Implementation (Briefly)}
\begin{itemize}
    \item Matlab Function block 사용 안함
    \item 기능 별로 모듈화 (for readability)
    \item Sensor data 만을 활용해서 차량을 제어함
    \item LKAS Module은 그대로 사용
    \begin{itemize}
        \item as black box
    \end{itemize}
\end{itemize}
\end{frame}

\begin{frame}
\frametitle{Demo Video}
\begin{center}
    \href{https://www.youtube.com/watch?v=jrDs9xtC62U}{Demo Video (Click)}
\end{center}
\end{frame}
\end{document}