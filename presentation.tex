\documentclass{beamer}
\usepackage{kotex}
\usepackage{amsmath}


\title{중간발표}
\author{장준민 \and 장지훈 \and 이예담}
\institute{Kookmin Univ. EE}
\date{\today}
\usetheme{Madrid}

\begin{document}
\frame{\titlepage}

\begin{frame}
\frametitle{Constraints}
    \begin{block}{About Vehicle}
        \setlength{\abovedisplayskip}{0pt}
        \setlength{\belowdisplayskip}{0pt}
        \begin{align*}
            (0.0 \leq VehicleSpeed \leq 5.5) \; &\wedge \\
            (InitialSpeed = 5.5) \; &\wedge \\
            (Camera \vee Lidar \vee Radar\vee Ultrasonic\vee ObjectCameara\vee World Viewer) \; &\wedge \\
        \end{align*}
    \end{block}
    \begin{block}{About Task}
        \setlength{\abovedisplayskip}{0pt}
        \setlength{\belowdisplayskip}{0pt}
        \begin{align*}
            \neg HitChild \; &\wedge \\ 
            StopAtSignal \; &\wedge \\
            CanLaneKeeping\; &\wedge \\
            CanPark \; &\wedge \\
        \end{align*}
    \end{block}
\end{frame}

\begin{frame}
\frametitle{Implementation (Briefly)}
\begin{itemize}
    \item Matlab function 사용 안함 (for performance)
    \begin{itemize}
        \item Matlab function 블럭이 시뮬레이션 동안 계속 interpreted 되는지, 중간에 한번 compile 되고 재사용 하는지(JIT)는 잘 모르겠음.
    \end{itemize}
    \item LKAS Module은 그대로 사용
    \begin{itemize}
        \item Black box 취급
    \end{itemize}
    \item 기능 별로 Module 화 시킴 (for readability)
    \item 요구사항대로, Sensor data 만을 활용해서 의사결정을 함.
\end{itemize}
\end{frame}

\begin{frame}
    
\end{frame}
\end{document}